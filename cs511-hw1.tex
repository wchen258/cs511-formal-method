%% to produce a PDF copy, issue the following command:
%%
%%     pdflatex propositional-logic-examples.tex
%%
%% in the same directory containing the LaTeX style files:
%%
%%     prooftree.sty  and  boxproof.sty

\documentclass[11pt,leqno,fleqn]{article}

\usepackage{graphicx} 
\usepackage{times}              % better fonts for mathematical symbols
\usepackage{bm}                 % unlike \boldmath,
                                % \bm can be used anywhere within math mode
\usepackage[scaled=0.9]{helvet} % makes text a little smaller throughout,
                                % but not the text in math mode.
\usepackage{prooftree}
\usepackage{boxproof}

\setlength\hoffset{-5pt}      % horizontal offset, to move text horizontally
\setlength{\textwidth}{4.5in} % try different widths
\setlength\voffset{-5pt}      % vertical offset, to move text vertically
\setlength{\textheight}{7in}  % try different heights

\newcommand{\Hide}[1]{}             % use \Hide{bla bla} to hide ``bla bla''
\newcommand{\code}[1]{\texttt{#1}}  % use \code{...} to produce ASCII chars
\newcommand{\Intro}[1]{{#1}{\textrm{i}}}
\newcommand{\Elim}[1]{{#1}{\textrm{e}}}

\title{CS511 HW1}
\author{Weifan Chen} 
\date{} % omit date

\begin{document}

\maketitle

\section{Coding Link}



\section{Exercise 1.1.2 (a)(b)(c)}
(a) $ ((\neg p)\land q) \rightarrow r$\\
(b) $ (p\rightarrow q) \land \neg ( (r\lor p) \rightarrow q) $\\
(c) $ (p\rightarrow q) \rightarrow (r \rightarrow (s \lor t))$\\

\section{Exercise 1.2.2. (a): $\neg p \to \neg q \vdash q \to p$}

\begin{proofbox}
\label{a1}\: \neg p \to \neg q \=\textrm{premise}\\
\[
	\label{a2}\: q \=\textrm{assumption}\\
	\label{a3}\: \neg\neg q \= \neg\neg \textrm{i}\ \ref{a2}\\
	\label{a4}\: \neg\neg p \= \textrm{MT \ref{a1},\ref{a3}}\\
	\label{a5}\: p \= \neg\neg e \ \ref{a4}\\
\]
\: q \to p \= \to i,\ref{a2},\ref{a5}
\end{proofbox}


\section{Exercise 1.2.2. (b): $ \neg p \lor \neg q \vdash \neg (p \land q)$}

\begin{proofbox}
\label{1}\: \neg p \lor \neg q \= \textrm{premise}\\
\[
	\label{2}\: \neg p \= \textrm{assumption} \\
	\[
		\label{3}\: p \land q \= \textrm{assumption}\\
		\label{4}\: p \= \Elim{\land}_1 \ \ref{3} \\
		\label{5}\: \bot \= \Intro{\bot}\ \ref{2}, \ref{4}\\
	\]
	\label{6}\: \neg (p\land q) \= \Intro{\neg}\ \ref{3} \textrm{-} \ref{5}\\
\]
\[
	\label{7}\: \neg q \= \textrm{assumption} \\
	\[
		\label{8}\: p \land q \= \textrm{assumption}\\
		\label{9}\: q \= \Elim{\land}_2 \ \ref{8} \\
		\label{10}\: \bot \= \Intro{\bot}\ \ref{7}, \ref{9}\\
	\]
	\label{11}\: \neg (p\land q) \= \Intro{\neg}\ \ref{8} \textrm{-} \ref{10}\\
\]
\: \neg (p \land q) \= \Elim{\lor}, \ref{1},\ref{2}\textrm{-}\ref{6},\ref{7}\textrm{-}\ref{11}\\
\end{proofbox}

\section{Exercise 1.2.2. (c) $\neg p , p\lor q \vdash q$}

\begin{proofbox}
\label{1}\: \neg q \= \textrm{premise}\\
\label{2}\: p \lor q \= \textrm{premise}\\
\[
	\label{3}\: q \= \textrm{assumption}\\
\]
\[
	\label{4}\: p \= \textrm{assumption}\\
	\label{5}\: \bot \= \Intro{\bot}\ \ref{1}, \ref{4}\\
	\label{6}\: q \= \Elim{\bot}\ \ref{5}\\
\]
\: q \= \Elim{\lor}\ \ref{2},\ref{3},\ref{4}\textrm{-}\ref{6}\\
\end{proofbox}

\end{document}

