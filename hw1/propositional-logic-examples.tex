%% to produce a PDF copy, issue the following command:
%%
%%     pdflatex propositional-logic-examples.tex
%%
%% in the same directory containing the LaTeX style files:
%%
%%     prooftree.sty  and  boxproof.sty

\documentclass[11pt,leqno,fleqn]{article}

\usepackage{graphicx} 
\usepackage{times}              % better fonts for mathematical symbols
\usepackage{bm}                 % unlike \boldmath,
                                % \bm can be used anywhere within math mode
\usepackage[scaled=0.9]{helvet} % makes text a little smaller throughout,
                                % but not the text in math mode.
\usepackage{prooftree}
\usepackage{boxproof}

\setlength\hoffset{-5pt}      % horizontal offset, to move text horizontally
\setlength{\textwidth}{4.5in} % try different widths
\setlength\voffset{-5pt}      % vertical offset, to move text vertically
\setlength{\textheight}{7in}  % try different heights

\newcommand{\Hide}[1]{}             % use \Hide{bla bla} to hide ``bla bla''
\newcommand{\code}[1]{\texttt{#1}}  % use \code{...} to produce ASCII chars
\newcommand{\Intro}[1]{{#1}{\textrm{i}}}
\newcommand{\Elim}[1]{{#1}{\textrm{e}}}

\title{Natural-Deduction Examples\\ in Propositional Logic}
\author{Assaf Kfoury} 
\date{} % omit date

\begin{document}

\maketitle

\section{Example: $P \vdash Q\to(P\land Q)$}

\begin{proofbox}
   \label{a1}\: P \= \textrm{premise} \\
   \[
      \label{a2}\: Q  \= \textrm{assume} \\
      \: P\land Q    \= \Intro{\land}\ \ref{a1},\ref{a2} \\
   \]
   \: Q\to(P\land Q) \= \Intro{\to}  \\
\end{proofbox}

\section{Example: $P\to(Q\to R) \vdash P\land Q\to R$}

\begin{proofbox}
   \label{b1}\: P\to(Q\to R) \= \textrm{premise} \\
   \[
      \label{b2}\: P\land Q    \= \textrm{assume} \\
      \label{b3}\: P \= {\Elim{\land}}_1\ \ \ref{b2}\\
      \label{b4}\: Q\to R \= \Elim{\to}\ \ \ref{b1},\ref{b3}\\
      \label{b5}\: Q \= {\Elim{\land}}_2\ \ \ref{b2}\\
      \: R \=\Elim{\to}\ \ \ref{b4},\ref{b5}\\
   \]
      \: P\land Q\to R \= \Intro{\to} \\
\end{proofbox}

\section{Example: $P\land Q\to R \vdash P\to(Q\to R)$}

\begin{proofbox}
   \label{c1}\: P\land Q\to R \= \textrm{premise} \\
   \[
      \label{c2}\: P     \= \textrm{assume} \\
      \[
         \label{c3}\: Q  \= \textrm{assume} \\
         \label{c4}\: P\land Q \= \Intro\land\ \ref{c2},\ref{c3}\\
         \: R \= \Elim\to\ \ref{c1},\ref{c4}
      \]
      \: Q\to R \= \Intro\to \\
   \]
   \: P\to(Q\to R) \= \Intro\to \\
\end{proofbox}

\section{Example: $P\to(Q\to R) \vdash (P\to Q)\to(P\to R)$}

\begin{proofbox}
   \label{d1}\: P\to(Q\to R) \= \textrm{premise} \\
   \[
      \label{d2}\: P\to Q  \= \textrm{assume} \\
      \[
         \label{d3}\: P    \= \textrm{assume} \\
         \label{d4}\: Q \= \Elim\to\ \ref{d2},\ref{d3}\\
         \label{d5}\: Q\to R \= \Elim\to \ref{d1},\ref{d3}\\
         \: R \= \Elim\to\ \ref{d5},\ref{d4}\\
      \]
      \: P\to R \= \Intro\to \\
   \]
   \: (P\to Q)\to(P\to R) \= \Intro\to \\
\end{proofbox}

\section{Example:  $\quad P\land\neg{Q}\to R,\ \ \neg{R},\ \ P \quad\vdash\quad Q$}

\begin{proofbox}
   \label{aa}\: P\land\neg{Q}\to R \= \textrm{premise}\\
   \label{bb}\: \neg{R} \= \textrm{premise}\\
   \label{cc}\: P \= \textrm{premise}\\
   \[
      \label{dd}\: \neg{Q} \= \textrm{assume}\\
      \label{ee}\: P\land\neg{Q} \= {\Intro{\land}}\ \ \ref{cc},\ref{dd}\\
      \label{ff}\: R \= \Elim{\to}\ \ \ref{aa},\ref{ee}\\
      \label{gg}\: \bot \= {\Elim{\neg{}}}\ \ \ref{ff},\ref{bb}\\
   \]
      \label{hh}\: \neg{\neg{Q}} \= \Intro{\neg{}} \\
      \: Q \= \Elim{\neg{\neg{}}}\ \ \ref{hh} \\
\end{proofbox}

\end{document}

